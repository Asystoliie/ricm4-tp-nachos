\documentclass[a4paper,10pt]{article}
\usepackage[utf8]{inputenc}
\usepackage{amssymb}
\usepackage{amsmath}
\usepackage{graphicx}
\newcommand{\HRule}{\rule{\linewidth}{0.5mm}}
\usepackage{subfigure}
\usepackage{listings}
\usepackage{multicol}
\usepackage[usenames,dvipsnames]{color}
\definecolor{light-gray}{gray}{0.95}
\lstset{language=C,basicstyle=\small,backgroundcolor=\color{light-gray}}
\usepackage[left=2.5cm, right=2.5cm]{geometry}
%top=tlength, bottom=blength,
%partie concernant la gestion des entêtes
\usepackage{fancyhdr}
\pagestyle{fancy}
\usepackage{lastpage}
\renewcommand\headrulewidth{1pt}
\fancyhead[L]{\textsc{TP1 DNS}}
\fancyhead[R]{\textsc{Polytech' Grenoble}}
\renewcommand\footrulewidth{1pt}
\fancyfoot[R]{ \textsc{RICM 4}}
%opening
\title{Etape 2}
\author{-}

\begin{document}

\maketitle

\section{Entrées-sorties asynchrones}

\subsection{Action 1}

  On execute la commande $./nachos-userprog -c$ et on voit que pour chaque lettre/mot écrit on a une copie et ainsi de suite. On observe le fichier
$userprog/ progtest.cc$, on observe l'utilisation des semaphores:

\begin{lstlisting}
  for (;;)
      {
	  readAvail->P ();	// wait for character to arrive
	  ch = console->GetChar ();
	  console->PutChar (ch);	// echo it!
	  writeDone->P ();	// wait for write to finish
	  if (ch == 'q')
	      return;		// if q, quit
      }
\end{lstlisting}


\subsection{Action 2}

  Afin qu'on puisse terminer la console avec un fin de fichier on ajoute dans la fonction $ConsoleTest$:

\begin{lstlisting}
  if (ch == EOF)
            return;
\end{lstlisting}

\subsection{Action 3}



\end{document}
