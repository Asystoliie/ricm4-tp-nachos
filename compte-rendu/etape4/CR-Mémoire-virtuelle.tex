\documentclass[a4paper,10pt]{article}
\usepackage[utf8]{inputenc}
\usepackage{amssymb}
\usepackage{amsmath}
\usepackage{graphicx}
\newcommand{\HRule}{\rule{\linewidth}{0.6mm}}
\usepackage{subfigure}
\usepackage{multicol}
\usepackage[usenames,dvipsnames]{color}
\definecolor{darkgray}{rgb}{0.95,0.95,0.95}
\usepackage{listings}
\usepackage{color}
\usepackage{textcomp}
\definecolor{listinggray}{gray}{0.9}
\definecolor{lbcolor}{rgb}{0.9,0.9,0.9}
\lstset{
    backgroundcolor=\color{lbcolor},
    tabsize=4,
    rulecolor=,
    language=C++,
    basicstyle=\scriptsize,
    upquote=true,
    aboveskip={1.5\baselineskip},
    columns=fixed,
    showstringspaces=false,
    extendedchars=true,
    breaklines=true,
    prebreak = \raisebox{0ex}[0ex][0ex]{\ensuremath{\hookleftarrow}},
    frame=single,
    showtabs=false,
    showspaces=false,
    showstringspaces=false,
    identifierstyle=\ttfamily,
    keywordstyle=\color[rgb]{0,0,1},
    commentstyle=\color[rgb]{0.133,0.545,0.133},
    stringstyle=\color[rgb]{0.627,0.126,0.941},
    backgroundcolor=\color{darkgray},
}

\usepackage[left=1.0cm, right=1.0cm, top=2cm, bottom=4cm]{geometry}
\usepackage{fancyhdr}
\pagestyle{fancy}
\usepackage{lastpage}
\renewcommand\headrulewidth{1pt}
\fancyhead[L]{\textsc{Nachos Étape 4 : Mémoire virtuelle}}
\fancyhead[R]{\textsc{Polytech' Grenoble}}
\renewcommand\footrulewidth{1pt}
\fancyfoot[R]{ \textsc{RICM 4}}

\begin{document}
\begin{titlepage}

\begin{center}


% Upper part of the page
\includegraphics[width=0.25\textwidth]{../images/logo}\\[1cm]

\textsc{\LARGE Polytech' Grenoble}\\[1.5cm]

\textsc{\Large RICM 4\`eme ann\'ee}\\[1.2cm]


% Title
\HRule \\[0.4cm]
{ \huge \bfseries NachOS\\[0.6cm]
Etape 4: Mémoire virtuelle}
\\[0.4cm]

\HRule \\[2cm]

% Author and supervisor
\begin{minipage}{0.4\textwidth}
\begin{flushleft} \large
\emph{\'Etudiants:}\\
Elizabeth \textsc{Paz} \\
Salem \textsc{Harrache}
\end{flushleft}
\end{minipage}
\begin{minipage}{0.4\textwidth}
\begin{flushright} \large
\emph{Enseignant:} \\
Vania \textsc{Marangozova}
\end{flushright}
\end{minipage}

\vfill

% Bottom of the page
{\large  Février 2012}

\end{center}

\end{titlepage}

\section{Git}

Pour voir les différences entre l'étape 3 et l'étape 4 vous pouvez lancer un
diff avec le tag step3 :
\begin{lstlisting}
git diff step3
\end{lstlisting}
ou alors directement avec le commit 78799c....
\begin{lstlisting}
git diff 78799ce7a7b788d0f3b501f0d85bab2f21c7190b
\end{lstlisting}

\section{Test de l'étape 4}

Dans notre test, on va créer trois processus, qui lancent chacun deux threads
qui vont ecrire leurs noms trois fois.

\begin{lstlisting}
Lancement du Test 1 :)
----------------------
./build-origin/nachos-userprog -rs 1 -x ./build/forkprocess
Debut du pere
...
Fin du pere
Debut du fils 0 : lancement des deux threads a et z
Debut du fils 1 : lancement des deux threads b et y
Debut du fils 2 : lancement des deux threads c et x
azbyxczaybxcza
Fin du thread main du fils 0
yb
Fin du thread main du fils 1
xc
Fin du thread main du fils 2
Machine halting!

Ticks: total 652352, idle 9380, system 75070, user 567902
Disk I/O: reads 0, writes 0
Console I/O: reads 0, writes 300
Paging: faults 0
Network I/O: packets received 0, sent 0

Cleaning up...
\end{lstlisting}

\section{Lecture dans la mémoire virtuelle}

Jusqu'a à présent, le (seul) procesus a son espace d'adresse à l'adresse zero. On ne
Chaque processus a son propre espaces


\section{Allocation des cadres de pages}

Nous avons légèrement modifié la class \textit{frameprovider}. En effet
la fonction, celle ci alloue un certain nombre de cadres de façon atomique. Un
processus à besoin de N cadres ou ne lance pas.

\lstset{caption={\textit{code/userprog/frameprovider.cc}}}
\begin{lstlisting}
int * FrameProvider::GetEmptyFrames(int n) {
    RandomInit(0);
    this->semFrameBitMap->P();
    int * frames = NULL;
    if (n <= this->bitmap->NumClear()) {
        frames = new int[n];
        for(int i=0; i<n; i++) {
            int frame = Random()%NumPhysPages;
            // Recherche d'une page libre
            while(this->bitmap->Test(frame)) {
                frame = Random()%NumPhysPages;
            }
            this->bitmap->Mark(frame);
            bzero(&(machine->mainMemory[ PageSize * frame ] ), PageSize );
            frames[i] = frame;
        }
    }
    this->semFrameBitMap->V();
    return frames;
}
\end{lstlisting}


L'utilisation dans \textbf{AddrSpace} est la suivante :

\lstset{caption={\textit{code/userprog/addrspace.cc}}}
\begin{lstlisting}
    int * frames = frameprovider->GetEmptyFrames((int) numPages);
    if (frames == NULL) {
        DEBUG ('p', "Pas suffisamment de memoire !\n");
        return;
    }

    // first, set up the translation
    pageTable = new TranslationEntry[numPages];
    for (i = 0; i < numPages; i++) {
        pageTable[i].virtualPage = i;
        // for now, virtual page # = phys page #
        pageTable[i].physicalPage = frames[i];
    [...]

    delete frames;
\end{lstlisting}

l'instruction return dans le constructeur avorte la construction de l'objet
AddrSpace, dans \textbf{\textit{do\_ForkExec}} il faut s'assurer que l'objet est
correctement intialisé.
On s'assure également que le destructeur libère les cadres un par un.


\section{Création d'un nouveau processus}

La création d'un nouveau processus se déroule en plusieurs etapes :
\begin{itemize}
\item Création d'un nouvel espace d'adressage
\item Creation d'un nouveau Thread main au quel on assoccie cette espace
d'adressage
\item Appel de Fork de ce nouveau thread.
\end{itemize}



\section{Terminaison}



\end{document}

